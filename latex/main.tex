\documentclass{article}

\title{MetaL --- A Library for Formalised Metatheory in Agda}
\author{Robin Adams}

\usepackage{amsmath}
\usepackage{amsthm}
\usepackage{agda}
\usepackage{catchfilebetweentags}
\usepackage{stmaryrd}
\usepackage{textgreek}

\DeclareUnicodeCharacter{8718}{\ensuremath{\qed}}
\DeclareUnicodeCharacter{8759}{\ensuremath{::}}
\DeclareUnicodeCharacter{10023}{\ensuremath{\lozenge}}
\DeclareUnicodeCharacter{10230}{\ensuremath{\longrightarrow}}

\newcommand{\eqdef}{\ensuremath{\stackrel{\mathrm{def}}{=}}}

\newtheorem{lm}{Lemma}[section]
\theoremstyle{definition}
\newtheorem{example}[lm]{Example}
\newtheorem{df}[lm]{Definition}

\begin{document}

\maketitle

\section{Introduction}

%TODO Cite PHOML paper

\subsection{Design Criteria}

This library was produced with the following design goals.

\begin{itemize}
\item
The library should be \emph{modular}.  There should be a type \AgdaDatatype{Grammar}, and results such as the Substitution Lemma 
%TODO Cross-ref
should be
provable 'once and for all' for all grammars.\footnote{For future versions of the library, we wish to have a type of reduction rules over a grammar, and a type of theories (sets of rules of deduction) over a grammar.}
\item
It should be possible for the user to define their own operations, such as path substitution %TODO Cross-ref
\item
Operations which are defined by induction on expressions should be definable by induction in Agda.  Results which are proved by induction on expressions should be proved by induction in Agda.
\end{itemize}

\section{Grammar}

\begin{example}[Simply Typed Lambda Calculus]
\label{ex:stlc}
For a running example, we will construct the grammar of the simply-typed lambda-calculus, with Church-typing and one constant ground type $\bot$.  On paper, in BNF-style, we write the grammar as follows:
\[ \begin{array}{lrcl}
\text{Type} & A & ::= & \bot \mid A \rightarrow A \\
\text{Term} & M & ::= & x \mid MM \mid \lambda x : A . M
\end{array} \]
\end{example}

\subsection{Taxonomy}

A \emph{taxonomy} is a set of \emph{expression kinds}, divided into \emph{variable kinds} and \emph{non-variable kinds}.
The intention is that the expressions of the grammar are divided into expression kinds.  Every
variable ranges over the expressions of one (and only one) variable kind.

\ExecuteMetaData[Grammar/Taxonomy.tex]{Taxonomy}

An \emph{alphabet} is a finite set of \emph{variables}, to each of which is associated a variable kind.  We write \AgdaDatatype{Var} \AgdaSymbol{V} \AgdaSymbol{K}
for the set of all variables in the alphabet \AgdaSymbol{V} of kind \AgdaSymbol{K}.

\ExecuteMetaData[Grammar/Taxonomy.tex]{Alphabet}
\ExecuteMetaData[Grammar/Taxonomy.tex]{Var}

\begin{example}
For the simply-typed lambda-calculus, there are two expression kinds: \AgdaFunction{type},
which is a non-variable kind, and \AgdaFunction{term}, which is a variable kind:

\ExecuteMetaData[STLC/Grammar.tex]{Taxonomy}
\end{example}

\subsection{Grammar}

\begin{df}
\label{df:kinds}
An \emph{abstraction kind} has the form $K_1 \rightarrow \cdots \rightarrow K_n \rightarrow L$,
where each $K_i$ is an abstraction kind, and $L$ is an expression kind.

A \emph{constructor kind} has the form $A_1 \rightarrow \cdots \rightarrow A_n \rightarrow K$,
where each $A_i$ is an abstraction kind, and $K$ is an expression kind.
\end{df}

To define these, we introduce the notion of a \emph{simple kind}:
a simple kind over sets $S$ and $T$ is an object of the form $s_1 \rightarrow \cdots \rightarrow s_n \rightarrow t$,
where each $s_i \in S$ and $t \in T$.

We implement this by saying a simple kind over $S$ and $T$ consists of a list of objects of $S$, and one object of $T$:

\ExecuteMetaData[Grammar/Taxonomy.tex]{SimpleKindA}

We can construct an object of type $\AgdaDatatype{SK} \, \AgdaBound{S} \, \AgdaBound{T}$ by writing 
$$ s_1 \longrightarrow \cdots \longrightarrow s_n \longrightarrow t \lozenge \enspace . $$
(The `$\lozenge$' symbol marks
the end of the simple kind.  It is needed to help Agda disambiguate the syntax.)

We are now able to write Definition \ref{df:kinds} like this:

\ExecuteMetaData[Grammar/Taxonomy.tex]{SimpleKindB}

A \emph{grammar} over a taxonomy consists of:
\begin{itemize}
\item
a set of \emph{constructors}, each with an associated constructor kind;
\item
a function assigning, to each variable kind, an expression kind, called its \emph{parent}.  (The intention is that, when a declaration $x : A$ occurs in a context, if $x$ has kind $K$, then the kind of $A$ is the parent of $K$.)
\end{itemize}

\ExecuteMetaData[Grammar/Base.tex]{Grammar}

\begin{df}
 We define simultaneously the set of \emph{expressions} of kind $K$ over $V$for every expression kind $K$ and
alphabet $V$;
and the set of \emph{abstractions} of kind $A$ over $V$ for every abstraction kind $A$ and alphabet
$V$.
\begin{itemize}
\item Every variable of kind $K$ in $V$ is an expression of kind $K$ over $V$.
\item If $c$ is a constructor of kind $A_1 \rightarrow \cdots \rightarrow A_n \rightarrow K$,
and $M_1$ is an abstraction of kind $A_1$, \ldots, $M_n$ is an abstraction of kind $A_n$ (all over $V$), then
$$ c M_1 \cdots M_n $$
is an expression of kind $K$ over $V$.
\item
An abstraction of kind $K_1 \rightarrow \cdots \rightarrow K_n \rightarrow L$ over $V$ is
an expression of the form
\[ [x_1, \ldots, x_n] M \]
where each $x_i$ is a variable of kind $K_i$, and $M$ is an expression of kind $L$ over
$V \cup \{ x_1, \ldots, x_n \}$.
\end{itemize}
\end{df}

In the Agda code, we define simultaneously the following four types:
\begin{itemize}
 \item $\AgdaFunction{Expression} \, \AgdaBound{V} \, \AgdaBound{K} = \AgdaDatatype{Subexp} \, \AgdaBound{V} \, \AgdaInductiveConstructor{-Expression} \, \AgdaBound{K}$,
the type of expressions of kind $K$;
\item $\AgdaFunction{VExpression} \, \AgdaBound{V} \, \AgdaBound{K} = \AgdaFunction{Expression} \,
\AgdaBound{V} \, (\AgdaInductiveConstructor{varKind} \, \AgdaBound{K})$, a convenient
shorthand when $K$ is a variable kind;
\item $\AgdaFunction{Abstraction} \, \AgdaBound{V} \, \AgdaBound{A}$, the type of abstractions
of kind $A$ over $V$
\item $\AgdaFunction{ListAbstraction} \, \AgdaBound{V} \, \AgdaBound{AA}$: if $\AgdaBound{AA} \equiv [ A_1, \ldots, A_n ]$,
then $\AgdaFunction{ListAbstraction} \, \AgdaBound{V} \, \AgdaBound{AA}$ is the type of lists of abstractions $[M_1, \ldots, M_n]$
such that each $M_i$ is of kind $A_i$.
\end{itemize}

\ExecuteMetaData[Grammar/Base.tex]{Expression}
  
\begin{example}
The grammar given in Example \ref{ex:stlc} has four constructors:
\begin{itemize}
 \item
$\bot$, of kind \AgdaFunction{type};
\item
$\rightarrow$, of kind $\AgdaFunction{type} \longrightarrow \AgdaFunction{type} \longrightarrow \AgdaFunction{type}$
\item
\AgdaFunction{appl}, of kind $\AgdaFunction{term} \longrightarrow \AgdaFunction{term} \longrightarrow \AgdaFunction{term}$
\item
$\lambda$, of kind $\AgdaFunction{type} \longrightarrow (\AgdaFunction{term} \longrightarrow \AgdaFunction{term}) \longrightarrow \AgdaFunction{term}$
\end{itemize}
The kind of the final constructor $\lambda$ should be read like this: $\lambda$ takes a type $A$
and a term $M$, binds a term variable $x$ within $M$, and returns a term $\lambda x:A.M$

\ExecuteMetaData[STLC/Grammar.tex]{Grammar}
\end{example}

\subsection{Families of Operations}

Our next aim is to define replacement and substitution.  Many of the results about these two operations have very similar proofs, so in order to avoid duplicating
code, we make the following definition.

\begin{df}[Family of Operations]
 A \emph{family of operations} $\Rightarrow$ consists of:
 \begin{itemize}
  \item for any alphabets $U$, $V$, a set $U \Rightarrow V$ of \emph{operations} from $U$ to $V$;
  \item for any operation $\sigma : U \Rightarrow V$ and variable $x : \AgdaDatatype{Var} \, U \, K$, an expression $\sigma(x) : \AgdaFunction{Expression} \,
  V \, K$ %TODO Check Var is an AgdaDatatype
  \item for any alphabet $V$ and variable kind $K$, an operation $\uparrow : V \Rightarrow V , K$
  \item for any alphabet $V$, an operation $1_V : V \Rightarrow V$
  \item for any operations $\rho : U \Rightarrow V$ and $\sigma : V \Rightarrow W$, an operation $\sigma \circ \rho : U \Rightarrow W$, the
  \emph{composition} of $\sigma$ and $\rho$;
  \item for any operation $\sigma : U \Rightarrow V$ and variable kind $K$, an operation $\sigma^\uparrow : U , K \Rightarrow V , K$, the \emph{lifting} of $\sigma$;
 \end{itemize}
such that:
\begin{itemize}
 \item $\uparrow(x) \equiv x$ for any variable $x$
 \item $1_V(x) \equiv x$ for any variable $x$
 \item $\sigma^\uparrow(x_0) \equiv x_0$ %TODO Check what notation we used earlier.
 \item $\sigma^\uparrow(x) \equiv \sigma(x) [ \uparrow ]$
 \item $(\sigma \circ \rho)(x) \equiv \rho(x) [ \sigma ]$
\end{itemize}
where, if $E : \AgdaFunction{Expression} \, U \, K$ and $\sigma : U \Rightarrow V$ then $E [ \sigma ] : \AgdaFunction{Expression} \, V \, K$,
the \emph{action} of $\sigma$ on $E$, is defined by
\begin{eqnarray*}
 x [ \sigma ] & \eqdef \sigma(x) \\
 ([x_1, \ldots, x_n] E) [ \sigma ] & \eqdef E [ \sigma^{\uparrow \uparrow \cdots \uparrow} ] \\
 (c E_1 \cdots E_n) [ \sigma ] & \eqdef c (E_1 [ \sigma ]) \cdots (E_n [ \sigma ])
\end{eqnarray*}

We write $\rho \sim \sigma$ iff $\rho$ and $\sigma$ are extensionally equal, i.e.~$\rho(x) \equiv \sigma(x)$ for every variable $x$.
\end{df}


The way that this is formalised in Agda is described in Appendix \ref{section:opfamilies}.

It is easy to see that our two examples of replacement and substitution fit this pattern.

\begin{df}[Replacement]
 \emph{Replacement} is the family of operations defined as follows.
 \begin{itemize}
  \item A \emph{replacement} from $U$ to $V$, $\rho : U \rightarrow_R V$, is a family of functions $\rho_K : \AgdaDatatype{Var} \, U \, K \rightarrow
  \AgdaDatatype{Var} \, V \, K$ for every variable kind $K$.
  \item For $x : \AgdaDatatype{Var} \, U \, K$, define $\rho(x) \eqdef \rho_K(x)$.
  \item Define $\uparrow : V \rightarrow_R V , K$ by $\uparrow_L(x) \equiv x$.
  \item Define $(1_V)_K(x) \equiv x$
  \item Define $(\sigma \circ \rho)_K (x) \equiv \sigma_K (\rho_K(x))$
  \item Define $\sigma^\uparrow_K(x_0) \equiv x_0$, and $\sigma^\uparrow_L(\uparrow x) \equiv \uparrow \sigma_L(x)$.
\end{itemize}
\end{df}

\ExecuteMetaData[Grammar/Replacement.tex]{REP}

We write $E \langle \rho \rangle$ for the action of a replacement $\rho$ on a subexpression $E$.

\begin{df}[Substitution]
 \emph{Substitution} is the family of operations defined as follows.
 \begin{itemize}
  \item A \emph{substitution} from $U$ to $V$, $\sigma : U \Rightarrow V$, is a family of functions $\sigma_K : \AgdaDatatype{Var} \, U \, K \rightarrow
  \AgdaDatatype{Expression} \, V \, K$ for every variable kind $K$.
  \item For $x : \AgdaDatatype{Var} \, U \, K$, define $\sigma(x) \eqdef \sigma_K(x)$
  \item Define $\uparrow : V \rightarrow_R V , K$ by $\uparrow_L(x) \equiv x$.
  \item Define $(1_V)_K(x) \equiv x$
  \item Define $(\sigma \circ \rho)_K(x) \equiv \rho_K(x) [ \sigma ]$
  \item Define $\sigma^\uparrow_K(x_0) \equiv x_0$ and $\sigma^\uparrow_L(\uparrow x) \equiv \sigma_L(x) \langle \uparrow \rangle$.
 \end{itemize}
\end{df}

\ExecuteMetaData[Grammar/Substitution/OpFamily.tex]{SUB}

We write $E [ \sigma ]$ for the action of a substitution $\sigma$ on a subexpression $E$.

\paragraph{Results about Families of Operations}

We can prove the following results about an arbitrary family of operations.

\begin{lm}
 \begin{enumerate}
  \item If $\rho \sim \sigma$ then $E [ \rho ] \equiv E [ \sigma ]$.
  \ExecuteMetaData[Grammar/OpFamily/Lifting.tex]{ap-congl}
  \item  $1_V^\uparrow = 1_{V, K}$
  \ExecuteMetaData[Grammar/OpFamily/LiftFamily.tex]{liftOp-idOp}
  \item $E [ 1_V ] \equiv E$
  \ExecuteMetaData[Grammar/OpFamily/LiftFamily.tex]{ap-idOp}
  \item $E [ \sigma \circ \rho ] \equiv E [ \rho ] [ \sigma ]$
  \ExecuteMetaData[Grammar/OpFamily/Composition.tex]{ap-comp}
  \item $\tau \circ (\sigma \circ \rho) \sim (\tau \circ \sigma) \circ \rho$
  \ExecuteMetaData[Grammar/OpFamily/OpFamily.tex]{assoc}
  \item If $\sigma : U \Rightarrow V$ then $1_V \circ \sigma \sim \sigma \sim \sigma \circ 1_U$
  \ExecuteMetaData[Grammar/OpFamily/OpFamily.tex]{unitl}
  \ExecuteMetaData[Grammar/OpFamily/OpFamily.tex]{unitr}
 \end{enumerate}
\end{lm}

\subsection{Substitution for the Last Variables}

Given an alphabet $V \cup \{ x_0, \ldots, x_n\}$ and expressions $E_0$, \ldots, $E_n$, we define the substitution
\[ [x_0 := E_0, \ldots, x_n := E_n] : V \cup \{ x_0, \ldots, x_n \} \Rightarrow V \]

\ExecuteMetaData[Grammar/Substitution/BotSub.tex]{botSub}

We have the following results about this substitution:
\begin{lm}
 \begin{enumerate}
  \item $E' \langle \uparrow \rangle [ x_0 := E ] \equiv E$
  \ExecuteMetaData[Grammar/Substitution/BotSub.tex]{botSub-up}
  \item $E' [ x_0 := E ] [ \sigma ] \equiv E' [ \sigma^\uparrow ] [ x_0:= E [ \sigma ] ]$
  \ExecuteMetaData[Grammar/Substitution/BotSub.tex]{comp-botSub}
 \end{enumerate}
\end{lm}

%TODO RepSub.agda

\section{Limitations}

\begin{itemize}
 \item There is no way to express that an expression depends on some variable kinds but not others.  (E.g. in our simply-typed lambda calculus example:
the types do not depend on the term variables.)  This leads to some boilerplate that is needed, proving lemmas of the form 

\begin{equation}
 \label{eq:boilerplate}
(\bot \AgdaBound{U}) [ \AgdaBound{σ} ] \equiv \bot \, \AgdaBound{V}
\end{equation}

There is a workaround for this special case.  We can declare all the types as constants:
%TODO Do this
This is what we used for the project PHOML.  %TODO Make consistent with rest of paper

For a general solution, we would need to parametrise alphabets by the set of variable kinds that may occur in them, and then prove results about mappings from one
type of alphabet to another.  We could then prove once-and-for-all versions of the lemmas like (\ref{eq:boilerplate}).  It remains to be seen whether this would still
be unwieldy in practice.
\end{itemize}

\appendix

\section{Formalisation of Families of Operations}
\label{section:opfamilies}

We define the type of families of operations in several stages, as follows.

\subsection{Pre-family of Operations}

\begin{df}[Pre-family of Operations]
%TODO Pick a better name
 A \emph{pre-family of operations} $\Rightarrow$ consists of:
 \begin{itemize}
  \item   \item for any alphabets $U$, $V$, a set $U \Rightarrow V$ of \emph{operations} from $U$ to $V$;
  \item for any operation $\sigma : U \Rightarrow V$ and variable $x : \AgdaDatatype{Var} \, U \, K$, an expression $\sigma(x) : \AgdaFunction{Expression} \,
  V \, K$ %TODO Check Var is an AgdaDatatype
  \item for any alphabet $V$ and variable kind $K$, an operation $\uparrow : V \Rightarrow V , K$
  \item for any alphabet $V$, an operation $1_V : V \Rightarrow V$
 \end{itemize}
 such that:
 \begin{itemize}
  \item $\uparrow(x) \equiv x$
  \item $1_V(x) \equiv x$
 \end{itemize}
\end{df}

\ExecuteMetaData[Grammar/OpFamily/PreOpFamily.tex]{PreOpFamily}

Let $\Rightarrow$ be a pre-family of operations.

\begin{df}
 Two operations $\rho, \sigma : U \Rightarrow V$ are \emph{extensionally equal}, $\rho \sim \sigma$, iff $\rho(x) \equiv \sigma(x)$ for all variables $x$.
\end{df}

We prove that this is an equivalence relation.

\ExecuteMetaData[Grammar/OpFamily/PreOpFamily.tex]{EqualOp}
%TODO Fix spacing

\subsection{Lifting}

\begin{df}[Lifting]
 A \emph{lifting} on a pre-family of operations is a mapping that, given an operation $\rho : U \Rightarrow V$ and variable kind $K$,
 returns an operation $\rho^\uparrow : U , K \Rightarrow V , K$.
\end{df}

\ExecuteMetaData[Grammar/OpFamily/Lifting.tex]{Lifting}

Given a pre-family of operations and a lifting, we can define the action of an operation on a subexpression:

\begin{eqnarray*}
 x [ \sigma ] & \eqdef \sigma(x) \\
 ([x_1, \ldots, x_n] E) [ \sigma ] & \eqdef E [ \sigma^{\uparrow \uparrow \cdots \uparrow} ] \\
 (c E_1 \cdots E_n) [ \sigma ] & \eqdef c (E_1 [ \sigma ]) \cdots (E_n [ \sigma ])
\end{eqnarray*}

\ExecuteMetaData[Grammar/OpFamily/Lifting.tex]{Action}

\subsection{Pre-family of Operations with Lifting}

\begin{df}[Pre-family of Operations with Lifting]
 A \emph{pre-family of operations with lifting} is given by a pre-family of operations $\Rightarrow$ and a lifting $^\uparrow$ such that:
 \begin{itemize}
  \item $\sigma^\uparrow(x_0) \equiv x_0$
  \item $\sigma^\uparrow(\uparrow x) \equiv \sigma(x) [ \uparrow ]$
 \end{itemize}
\end{df}

\ExecuteMetaData[Grammar/OpFamily/LiftFamily.tex]{LiftFamily}

\subsection{Composition}

\begin{df}[Composition]
 Let $\Rightarrow_1$, $\Rightarrow_2$ and $\Rightarrow_3$ be pre-families of operations with liftings.  A \emph{composition} $\circ : (\Rightarrow_1) ; (\Rightarrow_2) \rightarrow (\Rightarrow_3)$
 is a family of mappings
 \[ \circ_{UVW} : (V \Rightarrow_1 W) \times (U \Rightarrow_2 V) \rightarrow (U \Rightarrow_3 W) \]
 for all alphabets $U$, $V$, $W$ such that:
 \begin{align*}
  (\sigma \circ \rho)^\uparrow & \sim \sigma^\uparrow \circ \rho^\uparrow \\
  (\sigma \circ \rho)(x) & \equiv \rho(x) [ \sigma ]
 \end{align*}
for all $\rho$, $\sigma$, $x$.
\end{df}

\ExecuteMetaData[Grammar/OpFamily/Composition.tex]{Composition}

Let us write $[\ ]_1$, $[\ ]_2$, $[\ ]_3$ for the action of $\Rightarrow_1$, $\Rightarrow_2$, $\Rightarrow_3$ respectively.

\begin{lm}
 If $\circ$ is a composition, then $E[\sigma \circ \rho]_3 \equiv E [ \rho ]_2 [ \sigma ]_1$.
\end{lm}

\ExecuteMetaData[Grammar/OpFamily/Composition.tex]{ap-comp}

\begin{lm}
%TODO Ugly - how is this used?
 Let $\Rightarrow_1$, $\Rightarrow_2$, $\Rightarrow_3$, $\Rightarrow_4$ be pre-families of operations with liftings.  Suppose there exist compositions
 $(\Rightarrow_1); (\Rightarrow_2) \rightarrow (\Rightarrow_4)$ and $(\Rightarrow_2); (\Rightarrow_3) \rightarrow (\Rightarrow_4)$.  Let $\sigma : U \Rightarrow_2 V$.
 Suppose further that $E[\uparrow]_1 \equiv E[\uparrow]_2$ for all $E$.  Then
 \[ E [\uparrow]_3 [\sigma^\uparrow]_2 \equiv E [\sigma]_2 [\uparrow]_1 \]
 for all $E$.
\end{lm}

\ExecuteMetaData[Grammar/OpFamily/Composition.tex]{liftOp-up-mixed}

\begin{proof}
Let $\circ_1 : (\Rightarrow_1); (\Rightarrow_2) \rightarrow (\Rightarrow_4)$ and $\circ_2 : (\Rightarrow_2); (\Rightarrow_3) \rightarrow (\Rightarrow_4)$.
 We have $E[\uparrow]_3 [\sigma^\uparrow]_2 \equiv E [\sigma^\uparrow \circ_2 \uparrow]_4$ and $E [\sigma]_2 [\uparrow]_1 \equiv E [\uparrow \circ_1 \sigma]_4$,
 so it is sufficient to prove that $\sigma^\uparrow \circ_2 \uparrow \sim \uparrow \circ_1 \sigma$.
 
 We have
 \begin{align*}
  (\sigma^\uparrow \circ_2 \uparrow)(x) & \equiv \sigma^\uparrow(\uparrow(x)) \\
  & \equiv \sigma(x) [\uparrow]_2 \\
  & \equiv (\uparrow \circ_1 \sigma)(x)
 \end{align*}
\end{proof}

\subsection{Family of Operations}

\begin{df}[Family of Operations]
 A \emph{family of operations} consists of a pre-family with lifting $\Rightarrow$ and a composition $\circ : (\Rightarrow) ; (\Rightarrow) \rightarrow (\Rightarrow)$.
\end{df}

\ExecuteMetaData[Grammar/OpFamily/OpFamily.tex]{OpFamily}

\end{document}
