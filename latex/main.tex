\documentclass{article}

\title{MetaL --- A Library for Formalised Metatheory in Agda}
\author{Robin Adams}

\usepackage{amsmath}
\usepackage{agda}
\usepackage{todo}

\begin{document}

\maketitle

\section{Design Criteria}

This library was produced with the following design goals.

\begin{itemize}
\item
The library should be \emph{modular}.  There should be a type \AgdaDatatype{Grammar}, and results such as the Substitution Lemma \todo{Cross-ref} should be
provable 'once and for all' for all grammars.\footnote{For future versions of the library, we wish to have a type of reduction rules over a grammar, and a type of theories (sets of rules of deduction) over a grammar.}
\item
It should be possible for the user to define their own operations, such as path substitution \todo{Cross-ref}
\item
Operations which are defined by induction on expressions should be definable by induction in Agda.  Results which are proved by induction on expressions should be proved by induction in Agda.
\end{itemize}

\section{Grammar}

For a running example, we will construct the grammar of the simply-typed lambda-calculus, with Church-typing and one constant ground type $\bot$.  On paper, in BNF-style, we write the grammar as follows:
\[ \begin{array}{lrcl}
\text{Type} & A & ::= & \bot \mid A \rightarrow A \\
\text{Term} & M & ::= & x \mid MM \mid \lambda x : A . M
\end{array} \]

\subsection{Taxonomy}

A \emph{taxonomy} is a set of \emph{expression kinds}, divided into \emph{variable kinds} and \emph{non-variable kinds}.

\end{document}