\documentclass[envcountsame]{llncs}
\bibliographystyle{splncs03}

\title{MetaL --- A Library for Formalised Metatheory in Agda}
\author{Robin Adams\inst{1}}

\institute{University of Bergen \email{r.adams@ii.uib.no}}

\usepackage{amsmath}
%\usepackage{amsthm}
\usepackage{agda}
\usepackage{catchfilebetweentags}
\usepackage{stmaryrd}
\usepackage{textgreek}

\DeclareUnicodeCharacter{8718}{\ensuremath{\qed}}
\DeclareUnicodeCharacter{8759}{\ensuremath{::}}
\DeclareUnicodeCharacter{10023}{\ensuremath{\lozenge}}
\DeclareUnicodeCharacter{10230}{\ensuremath{\longrightarrow}}

\newcommand{\eqdef}{\ensuremath{\stackrel{\mathrm{def}}{=}}}

%\newtheorem{lemma}{Lemma}[section]
%\theoremstyle{definition}
%\newtheorem{definition}[lm]{Definition}

\begin{document}

\maketitle

\begin{abstract}
There are now many techniques for formalising metatheory (nominal sets, higher-order abstract syntax, etc.) but, in general, each requires the syntax and 
rules of deduction for a system to be defined afresh, and so all the proofs of basic lemmas must be written anew when we work with a new system, and
modified every time we modify the system. 

In this rough diamond, we present an early version of MetaL ("Metatheory Library"), a library for formalised metatheory in Agda. There is a type Grammar of 
grammars with binding, and types Red G of reduction relations and Rule G of sets of rules of deduction over G : Grammar. A grammar is given by a set of constructors, 
whose type specifies how many arguments it takes, and how many variables are bound in each argument. Reduction relations and rules of deduction are given by patterns,
or expressions involving second-order variables. 

The library includes a general proof of the substitution lemma. The final version is planned to include proofs of Church-Rosser for reductions with no critical pairs,
and Weakening and Substitution lemmas for appropriate sets of rules of deduction. 

MetaL has been designed with the following criteria in mind. It is easy to specify a grammar, reduction rule or set of rules of deduction: the Agda definition is the 
same length as the definition on paper. The general results are immediately applicable. When working within a grammar G, it should be possible to define functions by 
induction on expressions, and prove results by induction on expressions or induction on derivations, using only Agda's built-in pattern matching.
\end{abstract}


\section{Introduction}

%TODO Cite PHOML paper

\subsection{Design Criteria}

This library was produced with the following design goals.

\begin{itemize}
\item
The library should be \emph{modular}.  There should be a type \AgdaDatatype{Grammar}, and results such as the Substitution Lemma 
%TODO Cross-ref
should be
provable 'once and for all' for all grammars.\footnote{For future versions of the library, we wish to have a type of reduction rules over a grammar, and a type of theories (sets of rules of deduction) over a grammar.}
\item
It should be possible for the user to define their own operations, such as path substitution %TODO Cross-ref
\item
Operations which are defined by induction on expressions should be definable by induction in Agda.  Results which are proved by induction on expressions should be proved by induction in Agda.
\end{itemize}

\section{Grammar}

\begin{example}[Simply Typed Lambda Calculus]
\label{ex:stlc}
For a running example, we will construct the grammar of the simply-typed lambda-calculus, with Church-typing and one constant ground type $\bot$.  On paper, in BNF-style, we write the grammar as follows:
\[ \begin{array}{lrcl}
\text{Type} & A & ::= & \bot \mid A \rightarrow A \\
\text{Term} & M & ::= & x \mid MM \mid \lambda x : A . M
\end{array} \]
\end{example}

\subsection{Taxonomy}

A \emph{taxonomy} is a set of \emph{expression kinds}, divided into \emph{variable kinds} and \emph{non-variable kinds}.
The intention is that the expressions of the grammar are divided into expression kinds.  Every
variable ranges over the expressions of one (and only one) variable kind.

\ExecuteMetaData[Grammar/Taxonomy.tex]{Taxonomy}

An \emph{alphabet} is a finite set of \emph{variables}, to each of which is associated a variable kind.  We write \AgdaDatatype{Var} \AgdaSymbol{V} \AgdaSymbol{K}
for the set of all variables in the alphabet \AgdaSymbol{V} of kind \AgdaSymbol{K}.

\ExecuteMetaData[Grammar/Taxonomy.tex]{Alphabet}
\ExecuteMetaData[Grammar/Taxonomy.tex]{Var}

\subsection{Grammar}

\begin{definition}
\label{df:kinds}
An \emph{abstraction kind} has the form $K_1 \rightarrow \cdots \rightarrow K_n \rightarrow L$,
where each $K_i$ is an abstraction kind, and $L$ is an expression kind.

A \emph{constructor kind} has the form $A_1 \rightarrow \cdots \rightarrow A_n \rightarrow K$,
where each $A_i$ is an abstraction kind, and $K$ is an expression kind.
\end{definition}

A \emph{grammar} over a taxonomy consists of:
\begin{itemize}
\item
a set of \emph{constructors}, each with an associated constructor kind;
\item
a function assigning, to each variable kind, an expression kind, called its \emph{parent}.  (The intention is that, when a declaration $x : A$ occurs in a context, if $x$ has kind $K$, then the kind of $A$ is the parent of $K$.)
\end{itemize}

\ExecuteMetaData[Grammar/Base.tex]{Grammar}

\begin{definition}
 We define simultaneously the set of \emph{expressions} of kind $K$ over $V$for every expression kind $K$ and
alphabet $V$;
and the set of \emph{abstractions} of kind $A$ over $V$ for every abstraction kind $A$ and alphabet
$V$.
\begin{itemize}
\item Every variable of kind $K$ in $V$ is an expression of kind $K$ over $V$.
\item If $c$ is a constructor of kind $A_1 \rightarrow \cdots \rightarrow A_n \rightarrow K$,
and $M_1$ is an abstraction of kind $A_1$, \ldots, $M_n$ is an abstraction of kind $A_n$ (all over $V$), then
$$ c M_1 \cdots M_n $$
is an expression of kind $K$ over $V$.
\item
An abstraction of kind $K_1 \rightarrow \cdots \rightarrow K_n \rightarrow L$ over $V$ is
an expression of the form
\[ [x_1, \ldots, x_n] M \]
where each $x_i$ is a variable of kind $K_i$, and $M$ is an expression of kind $L$ over
$V \cup \{ x_1, \ldots, x_n \}$.
\end{itemize}
\end{definition}

In the Agda code, we define simultaneously the following four types:
\begin{itemize}
 \item $\AgdaFunction{Expression} \, \AgdaBound{V} \, \AgdaBound{K} = \AgdaDatatype{Subexp} \, \AgdaBound{V} \, \AgdaInductiveConstructor{-Expression} \, \AgdaBound{K}$,
the type of expressions of kind $K$;
\item $\AgdaFunction{VExpression} \, \AgdaBound{V} \, \AgdaBound{K} = \AgdaFunction{Expression} \,
\AgdaBound{V} \, (\AgdaInductiveConstructor{varKind} \, \AgdaBound{K})$, a convenient
shorthand when $K$ is a variable kind;
\item $\AgdaFunction{Abstraction} \, \AgdaBound{V} \, \AgdaBound{A}$, the type of abstractions
of kind $A$ over $V$
\item $\AgdaFunction{ListAbstraction} \, \AgdaBound{V} \, \AgdaBound{AA}$: if $\AgdaBound{AA} \equiv [ A_1, \ldots, A_n ]$,
then $\AgdaFunction{ListAbstraction} \, \AgdaBound{V} \, \AgdaBound{AA}$ is the type of lists of abstractions $[M_1, \ldots, M_n]$
such that each $M_i$ is of kind $A_i$.
\end{itemize}

\ExecuteMetaData[Grammar/Base.tex]{Expression}
  
\begin{example}
The grammar given in Example \ref{ex:stlc} has four constructors:
\begin{itemize}
 \item
$\bot$, of kind \AgdaFunction{type};
\item
$\rightarrow$, of kind $\AgdaFunction{type} \longrightarrow \AgdaFunction{type} \longrightarrow \AgdaFunction{type}$
\item
\AgdaFunction{appl}, of kind $\AgdaFunction{term} \longrightarrow \AgdaFunction{term} \longrightarrow \AgdaFunction{term}$
\item
$\lambda$, of kind $\AgdaFunction{type} \longrightarrow (\AgdaFunction{term} \longrightarrow \AgdaFunction{term}) \longrightarrow \AgdaFunction{term}$
\end{itemize}
The kind of the final constructor $\lambda$ should be read like this: $\lambda$ takes a type $A$
and a term $M$, binds a term variable $x$ within $M$, and returns a term $\lambda x:A.M$

\ExecuteMetaData[STLC/Grammar.tex]{Grammar}
\end{example}

\subsection{Families of Operations}

Our next aim is to define replacement and substitution.  

\begin{definition}[Replacement]
 \begin{itemize}
  \item A \emph{replacement} from $U$ to $V$, $\rho : U \rightarrow_R V$, is a family of functions $\rho_K : \AgdaDatatype{Var} \, U \, K \rightarrow
  \AgdaDatatype{Var} \, V \, K$ for every variable kind $K$.
  \item For $x : \AgdaDatatype{Var} \, U \, K$, define $\rho(x) \eqdef \rho_K(x)$.
  \item Define $\uparrow : V \rightarrow_R V , K$ by $\uparrow_L(x) \equiv x$.
  \item Define $(1_V)_K(x) \equiv x$
  \item Define $(\sigma \circ \rho)_K (x) \equiv \sigma_K (\rho_K(x))$
  \item Define $\sigma^\uparrow_K(x_0) \equiv x_0$, and $\sigma^\uparrow_L(\uparrow x) \equiv \uparrow \sigma_L(x)$.
\end{itemize}
\end{definition}

We write $E \langle \rho \rangle$ for the action of a replacement $\rho$ on a subexpression $E$.

\begin{definition}[Substitution]
 \begin{itemize}
  \item A \emph{substitution} from $U$ to $V$, $\sigma : U \Rightarrow V$, is a family of functions $\sigma_K : \AgdaDatatype{Var} \, U \, K \rightarrow
  \AgdaDatatype{Expression} \, V \, K$ for every variable kind $K$.
  \item For $x : \AgdaDatatype{Var} \, U \, K$, define $\sigma(x) \eqdef \sigma_K(x)$
  \item Define $\uparrow : V \rightarrow_R V , K$ by $\uparrow_L(x) \equiv x$.
  \item Define $(1_V)_K(x) \equiv x$
  \item Define $(\sigma \circ \rho)_K(x) \equiv \rho_K(x) [ \sigma ]$
  \item Define $\sigma^\uparrow_K(x_0) \equiv x_0$ and $\sigma^\uparrow_L(\uparrow x) \equiv \sigma_L(x) \langle \uparrow \rangle$.
 \end{itemize}
\end{definition}

We write $E [ \sigma ]$ for the action of a substitution $\sigma$ on a subexpression $E$.

%\subsubsection{Results about Replacement and Substitution.}

We can prove the following results about substitution, and analagous results about replacement.
\begin{lemma}
 \begin{enumerate}
  \item If $\rho \sim \sigma$ then $E [ \rho ] \equiv E [ \sigma ]$.
%  \ExecuteMetaData[Grammar/OpFamily/Lifting.tex]{ap-congl}
  \item  $1_V^\uparrow = 1_{V, K}$
%  \ExecuteMetaData[Grammar/OpFamily/LiftFamily.tex]{liftOp-idOp}
  \item $E [ 1_V ] \equiv E$
%  \ExecuteMetaData[Grammar/OpFamily/LiftFamily.tex]{ap-idOp}
  \item $E [ \sigma \circ \rho ] \equiv E [ \rho ] [ \sigma ]$
%  \ExecuteMetaData[Grammar/OpFamily/Composition.tex]{ap-comp}
  \item $\tau \circ (\sigma \circ \rho) \sim (\tau \circ \sigma) \circ \rho$
%  \ExecuteMetaData[Grammar/OpFamily/OpFamily.tex]{assoc}
  \item If $\sigma : U \Rightarrow V$ then $1_V \circ \sigma \sim \sigma \sim \sigma \circ 1_U$
%  \ExecuteMetaData[Grammar/OpFamily/OpFamily.tex]{unitl}
%  \ExecuteMetaData[Grammar/OpFamily/OpFamily.tex]{unitr}
 \end{enumerate}
\end{lemma}

% \subsection{Substitution for the Last Variables}
% 
% Given an alphabet $V \cup \{ x_0, \ldots, x_n\}$ and expressions $E_0$, \ldots, $E_n$, we define the substitution
% \[ [x_0 := E_0, \ldots, x_n := E_n] : V \cup \{ x_0, \ldots, x_n \} \Rightarrow V \]
% 
% \ExecuteMetaData[Grammar/Substitution/BotSub.tex]{botSub}
% 
% We have the following results about this substitution:
% \begin{lemma}
%  \begin{enumerate}
%   \item $E' \langle \uparrow \rangle [ x_0 := E ] \equiv E$
%   \ExecuteMetaData[Grammar/Substitution/BotSub.tex]{botSub-up}
%   \item $E' [ x_0 := E ] [ \sigma ] \equiv E' [ \sigma^\uparrow ] [ x_0:= E [ \sigma ] ]$
%   \ExecuteMetaData[Grammar/Substitution/BotSub.tex]{comp-botSub}
%  \end{enumerate}
% \end{lemma}
% 
% %TODO RepSub.agda

\section{Reduction Relations}

\section{Rules of Deduction}

\section{Limitations}

\begin{itemize}
 \item There is no way to express that an expression depends on some variable kinds but not others.  (E.g. in our simply-typed lambda calculus example:
the types do not depend on the term variables.)  This leads to some boilerplate that is needed, proving lemmas of the form 

\begin{equation}
 \label{eq:boilerplate}
(\bot \AgdaBound{U}) [ \AgdaBound{σ} ] \equiv \bot \, \AgdaBound{V}
\end{equation}

There is a workaround for this special case.  We can declare all the types as constants:
%TODO Do this
This is what we used for the project PHOML. %TODO Cite  %TODO Make consistent with rest of paper

For a general solution, we would need to parametrise alphabets by the set of variable kinds that may occur in them, and then prove results about mappings from one
type of alphabet to another.  We could then prove once-and-for-all versions of the lemmas like (\ref{eq:boilerplate}).  It remains to be seen whether this would still
be unwieldy in practice.
\end{itemize}

\section{Related Work}

\section{Conclusion}

\bibliography{type}

\end{document}
